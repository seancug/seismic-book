\chapter{SU信号处理}
\section{1D滤波操作}
地震处理中很大一部分工作可以称为滤波。在SU中1D滤波应用程序可以进行简单的滤波操作,和更复杂的反褶积和子波整形处理。这些操作成为1D,是因为这些操作处理数据是一道一道进行的。\\
SU包含的命令有:\\
\begin{tabular}{ll}
	\toprule
	sufilter & 应用零相位正弦平方斜坡滤波\\	
	subfilt & 应用Butterworth带通滤波	\\
	suacor & 自相关	\\
	suconv, suxcor & 利用用户提供的滤波器进行褶积和相关	\\
	supef & Wiener预测误差滤波\\	
	sushape & Wiener整形滤波\\	
	suresamp & 时间域重采样	\\
	sufrac & 采用一般的时间微分、积分、相位移动,注:时间域数据\\	
	sumedian & 沿着用户确定的多边形曲线,根据道头字指定的与曲线的距离,进行中值滤波\\	
	sutvband & 时变带通滤波(正弦平方滤波)\\
	\bottomrule
\end{tabular}

\subsection{sufilter - 应用零相位正弦平方斜坡滤波}
sufiletr提供了可用于一般的带通、带陷、低通、高通和陷波滤波,使用sufilter进行滤波的例子如下:
\begin{lstlisting}
suplane | sufilter f=10,20,30,60 amps=0,1,1,0 | suxwigb title="10,20,30,60 hz bandpass" 
suplane | sufilter f=10,20,30,60 amps=1,0,0,1 | suxwigb title="10,20,30,60 hz bandreject" &
suplane | sufilter f=10,20,30,60 amps=1,1,0,0 | suxwigb title="10,20 hz lowpass" &
suplane | sufilter f=50,60,70 amps=1,0,1 | suxwigb title="60 hz notch" &
\end{lstlisting}

\subsection{subfilt - 应用Butterworth带通滤波 }
subfilt与程序sufilter工程相似,它使用Butterworth滤波器进行滤波:
\begin{lstlisting}
suplane | subfilt fstoplo=10 fpasslo=20 fpasshi=30 fstophi=60 | suxwigb title="10,20,30,60 hz bandpass” 
\end{lstlisting}

\subsection{suconv, suxcor - 使用用户设计的滤波器进行褶积和相关}
褶积和互相关可分别使用suconv和suxcor命令实现。滤波器可以通过在命令行输入向量、或者通过含有某个道的SU格式文件提供。\par
下面是一个可控联系扫描信号进行相关处理的例子。通过联合使用suvibro、suplane和suconv,产生类似可控连续震动的数据。生成可控连续震动的命令如下:
\begin{lstlisting}
suvibro > junk.vib.su
suplane | suconv sufile=junk.vib.su > plane.vib.su
\end{lstlisting}
使用surange命令查看junk.vib.su数据的信息:
\begin{lstlisting}
surange < junk.vib.su
1 traces:
tracl=1 ns=2500 dt=4000 sfs=10 sfe=60
slen=10000 styp=1
\end{lstlisting}
显示可控震源扫描信号有2500个采样点,因此做下面所示的相关分析:
\begin{lstlisting}
suxcor < plane.vib.su sufile=junk.vib.su | suwind itmin=2500 itmax=2563 | sushw key=delrt a=0.0 > data.su
\end{lstlisting}
参数itmin=sweeplength和itmax=sweeplength+nsout,参数nsout是期望输出的采样点数,最后使用sushw设置道延迟为0。选择参数itmin=sweeplength将保证数据从正确的值起始。

\subsection{supef - Wiener预测误差滤波}
\subsection{sushape - Wiener整形滤波}

\section{2D滤波操作}
二维空间$(k1,k2)$和$(F,K)$域滤波有助于改变数据的倾角信息。下面的这些程序初步的提供了一套波数$(K)$域和频率-波数$(F,K)$域滤波操作命令:\par
\begin{tabular}{lp{0.7\textwidth}}
	\toprule
	sukfilter & radially symmetric K-domain, $sin^2$-tapered, polygonal filter\\
	suk1k2filter & symmetric box-like K-domain filter defined by the cartesian product of two $sin^2$-tapered polygonal filters defined in k1 and k2\\
	sukfrac  & apply FRACtional powers of i-k to data, with phase shift\\
	sudipfilt & DIP-or better-SLOPE Filter in f-k domain\\
	\bottomrule
\end{tabular}
\section{Fourier变换}
在SU软件包中有1D和2D Fourier变换程序。\par 
2D变换包括地震$F-K$变换,假定输入数据的块空间是时间的,第二维是空间的。\par
对非地震$K1-K2$变换,假定输入是纯二维空间$(x1, x2)$数据。
\subsection{1D Fourier变换}
\begin{tabular}{ll}
	\toprule
	sufft & 从实数时间道变换到复数频率道(正变换)\\
	suifft & 从复数频率道变换到实数时间道(逆变换)\\	
	suamp & 输出振幅、相位、实部和虚部(从frequency, x)\\	
	suspecfx & 地震数据Fourier频谱分析(时间T到频率F)\\
	\bottomrule
\end{tabular}\\
例如:
\begin{lstlisting}
suplane | suxwigb title="Original Data" 
suplane | sufft | suifft | sushw key=d1,dt a=0,4000 | suxwigb
\end{lstlisting}
结果与输入完全一样,除了地震道结果中有更多的采样点,因为变换时要补零。\\
要浏览sufft程序输出的振幅谱和相位谱、实部和虚部,操作如下:\\
\begin{lstlisting}
suplane | sufft | suamp mode=amp | suxwigb title="amplitude" 
suplane | sufft | suamp mode=phase | suxwigb title="phases" 
suplane | sufft | suamp mode=real | suxwigb title="real" 
suplane | sufft | suamp mode=imag | suxwigb title="imaginary" 
\end{lstlisting}
SU数据格式可以存储复数数据的实部和虚部。键入下面的命令可以查看格式的道头设置位置:
\begin{lstlisting}[frame=single]
suplane | sufft | surange
sufft: d1=3.571428
32 traces:
tracl=(1,32) tracr=(1,32) trid=11 offset=400 ns=72
dt=4000 d1=3.571428
\end{lstlisting}
显示出道头字trid=11,显示出FFT变换输出数据是如何排列的.
当然,绝大多数时候,我们只想快速看看一个地震道或一块地震数据的振幅谱,这时可以用suspecfx命令:
\begin{lstlisting}
suplane | suspecfx | suximage title="F-X Amplitude Spectrum" 
\end{lstlisting}
这将直接显示输入SU数据每道的振幅谱。

\subsection{2D Fourier变换}
如果数据实际上为(时间、空间)坐标数据,那么2D Fourier变换输出的就是F-K(频率、波数)域数据。如果考虑两个空间坐标(x1,x2)数据,进行2D Fourier变换,这时输出就是(k1,k2)2D波数域数据。\par\mvspace
\begin{tabular}{ll}
	\toprule
	SUSPECFK & 数据的F-K域Fourier频谱分析\\
	SUSPECK1K2 &  $(x1,x2)$数据2D$(K1-K2)$Fourier频谱分析\\
	\bottomrule
\end{tabular}\\\\
例子:
\begin{lstlisting}
suplane | suspecfk | suximage title="F-K Amplitude Spectrum" 
suplane | suspeck1k2 | suximage title="K1-K2 Amplitude Spectrum" 
\end{lstlisting}

\section{Hilbert变换、道属性和时间-频率域}
\begin{tabular}{lp{0.7\textwidth}}
	\toprule
	suhilb & Hilbert变换\\
	suattrbutes & 道属性(瞬时振幅、相位和频率)\\
	sugabor & 通过Gabor变换(类似多参数滤波分析),输出地震数据时间-频率结果\\
	\bottomrule
\end{tabular}\\\\
进行Hilbert变换的例子如下:
\begin{lstlisting}
suplane | suhilb | suxwigb title="Hilbert Transform"
\end{lstlisting}
查看suattributes命令生成道属性,将相应的显示出瞬时振幅、相位和频率:
\begin{lstlisting}
suvibro | suxgraph title="Vibroseis sweep" 
suvibro | suattributes mode=amp | suxgraph title="Inst. amplitude" 
suvibro | suattributes mode=phase unwrap=1.0 | suxgraph title="Inst. phase" 
suvibro | suattributes mode=freq | suxgraph title="Inst. frequency" 
\end{lstlisting}
使用sugabor命令可以进行时间-频率域分析:
\begin{lstlisting}
suvibro | sugabor | suximage title="time frequency plot" 
\end{lstlisting}

\section{Radon变换  Tau\_P滤波}
Radon变换或“Tau\_P”变换时压制多次波和进行数据改造的有效方法,涉及的程序有:\\
\begin{tabular}{lp{0.7\textwidth}}
	\toprule
	sutaup & 正向和逆向(forwared and inverse)T-X和F-K全局倾斜叠加\\
	suharlan & 信号噪声分离,应用Harlan可逆线性变换方法(1984)\\	
	suradon & 计算正或逆Radon变换,或用抛物线Radon变换估计并去除多次波	\\
	suinterp & 使用自动同相轴拾取方法进行道插值\\
	\bottomrule
\end{tabular}\\\\
相关使用方法,请参阅:\$CWPROOT/src/demos/Tau\_P下的例子

\section{suresamp - 数据时间域重采样}
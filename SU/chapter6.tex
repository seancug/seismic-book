\chapter{demos - 演示实例}
目录\$CWPROOT/src/demos中有多个子目录,用于运行Shell脚本来演示SU命令的使用。
\begin{enumerate}
	\item /3D\_Data\_Viewing 三维数据的显示
	\item /3D\_Model\_Building 构建3D模型\\
	- Tetramod:建立地震波速度模型编码演示程序“tetra”.	
	\item  /Amplitude\_Correction : 振幅修正\\
	- sudipdivcor: performs the conventional and dip-dependent divergence corrections
	- sudivcor: performs the conventional divergence corrections.
	\item  /BC\_Examples:BC例子
	\item  /Block:免费软件包uncert中的”block”程序演示
	\item  /Datuming:基准面例子\\	
	- Finite\_Difference: 传统的有限差分基准面\\
	- Kirchhoff\_Datuming: Kirchhoff基准面.
	\item  /Deconvolution:反褶积演示\\
	- FX: 频率波数域滤波\\
	- Wiener\_Levinson: 预测误差滤波.
	\item  Dip\_Moveout:倾角校正\\
	- Converted\_Wave: performs  Converted Wave Dip Moveout ondata sorted in to CDP gathers\\
	- Fti\_Media: 预测误差滤波\\
	- Ti\_Media: 横向均匀介质倾角校正
	\item  Filtering:滤波演示\\
	- Subfilt: Butterworth滤波\\
	- Sudipfilt: 倾角滤波\\
	- Sufilter: 多边形零相位频率滤波\\
	- Sugabor: gabor变换频率-时间滤波\\
	- Suk1k2filter: 波数滤波(箱形,box)\\
	- Sukfilter: 波数滤波(环形,annular)\\
	- sumedian: 中值滤波
	\item  Header\_Remapping:观测系统坐标转换
	\item  Kaperture
	\item  Making\_Data:生成合成地震记录\\
	- CommonOffset: 共偏移距和零偏移距道集\\
	- ShotGathers: 共炮点道集
	\item Migration\_Inversion:偏移(反演)\\
	- Post\_stack: 叠后偏移演示\\
	- Pre\_stack: 叠前偏移演示\\
	- Slant\_Stack: slant stack 偏移演示
	\item Muting:切除演示
	\item NMO:传统正常时差校正和叠加演示
	\item Noise:加噪音演示
	\item Offset\_Continuation:小偏移距共偏移距延拓
	\item Picking:道拾取\\
	- Supickamp: 自动拾取
	\item Plotting:绘图演示
	\item Ray\_Tracing:射线追踪\\
	- Rayt2d: \\
	- Wkbj: 逆风有限差分法求解程函方程
	\item Reading\_Data:segd数据的读取\\
	- segdread: 
	\item Reflection\_Coefficients:反射系数\\
	- Linrort: \\
	- Refaniso: 
	\item Refraction:折射波应用\\
	- GRM:单层广义互逆法(Generalized Reciprocal Method for a single layer)
	\item Selecting\_Traces:用道头抽道做特殊处理
	\item Sorting\_Traces:用susort命令做道分选\\
	- Demo: \\
	- Susorty: \\
	- Tutorial: 
	\item Stacking\_Traces:道集叠加\\
	- Diversity\_Stacking: \\
	- Phase\_Weighted\_Stacking: 
	\item Statics:静校正\\
	- Residual\_Refraction\_Statics:\\ 
	- Residual\_Statics: 
	\item SUManual:SU用户手册中Shell脚本绘图命令\\
	- plotting: 绘图演示
	\item Synthetic:制作合成记录\\
	- Finite Difference: 有限差分2D模拟\\
	- From\_Well\_Logs:	\\
	- Impulse: 脉冲子波	\\
	- Kirchhoff: Kirchhoff模拟\\	
	- Reectivity: 发射模拟	\\
	- Suaddevent:make null traces with sunull, add linear moveout arrivals with suaddevent	\\
	- Suwaveform:wavelet generation with SUWAVEFORM	\\
	- Tetra: 四面形模拟\\
	- Tri: 三角形模拟	\\
	- Trielas:2D raytracing program for transversely isotropic media with in-plane rotated axis of symmetry	\\
	- CSHOT: 查看 \$CWPROOT/src/Fortran/CSHOT
	\item Tau\_P:Tau\_P或倾斜叠加滤波\\
	- Suharlan: Bill Harlan倾斜叠加噪声压制算法\\
	- Suradon: \\
	- Sutaup: 传统的Tau\_P变换
	\item Three\_Component\_Traces:三分量道集\\
	- Alford\_Rotation: \\
	- Polarization\_Analysis: 
	\item Time\_Freq\_Analysis:时间-频率域分析
	\item Trace\_Headers\_and\_Values\\
	- Suascii: \\
	- Sucountkey: \\
	- Sudumptrace: \\
	- Sukeycount: 
	\item Utilities:其它工具\\
	- Sucommand: 
	\item Velocity\_Analysis:基于NMO速度分析\\
	- Residual\_Moveout\_Analysis: \\
	- Ti\_Media: \\
	- Traditional: 	\\
	- Velancc: 
	\item Velocity\_Profiles:制作速度剖面\\
	- Makevel: \\
	- Marmousi: \\
	- Unif2: \\
	- Unisam2: \\
	- Vel2stiff: \\
	- Velconv: 
	\item Vibroseis\_Sweeps:用suvibro程序生成可控震源扫描输出道
	\item Wavelet\_Transforms\\
	- Mrafxzwt: Multi-Resolution Analysis (MRA) of a function F(X,Z) by Wavelet Transform
	\item Well\_Log:测井数据
	- Las2su: convert a LAS file to SU format
	\item 建议新手按下面的顺序进行演示学习:
	Making\_Data目录中含有用susynlv程序制作合成记录跑集和共偏移道集的演示流程。尤其是要注意演示中好的显示标注。\par
	Filtering目录中包含一些实际数据处理中消除地滚波和初至的演示程序。\par
	Deconvolution目录中演示了使用supef和其他工具简单合成脉冲道集,来举例说明去混响和脉冲反褶积处理。演示程序中包含使用loops系统检验滤波参数影响的命令。\par
	Sorting\_Traces是一个交互的脚本,加强了文献中讨论的一些Unix和SU基本知识。
\end{enumerate}
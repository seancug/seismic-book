\chapter{实际地震资料处理}

\section{道充零 sukill} 
\begin{lstlisting}
sukill < $indata min=61 count=2
\end{lstlisting}
道充零没有关键字,通过数道得到他们。

\section{f-x显示}
suspecf制作fx频谱图,suop op=norm归一化操作
\begin{lstlisting}
# Create frequency spectrum of each trace and normalize
suspecfx < $indata | suop op=norm |
# Plot frequency traces
suximage xbox=10 ybox=10 wbox=400 hbox=600 label1=" Freq (Hz)" windowtitle="f-x spectrum" title=$indata cmap=hsv7 legend=1 unit=Amplitude verbose=0 bclip=0.5 wclip=0.0 
\end{lstlisting}
交互脚本见附件fxdisp.sh

\section{滤波、重采样抽道集}
滤波sufilter例子:
\begin{lstlisting}
#Examples of filters:							
Bandpass:   sufilter <data f=10,20,40,50 | ...			
Bandreject: sufilter <data f=10,20,30,40 amps=1.,0.,0.,1. | ..	
Lowpass:    sufilter <data f=10,20,40,50 amps=1.,1.,0.,0. | ...	
Highpass:   sufilter <data f=10,20,40,50 amps=0.,0.,1.,1. | ...	
Notch:      sufilter <data f=10,12.5,35,50,60 amps=1.,.5,0.,.5,1. |..	

sufilter < $indata f=16,21,85,95 amps=0,1,1,0 | 
suresamp nt=2750 dt=.004 | susort > $outdata cdp offset
\end{lstlisting}

\section{球面补偿}
球面补偿简单的应用,已分贝形式查看比较补偿前后的变化。:
\begin{lstlisting}
sugain < indata.su tpow=2 > outdata.su
suwind < $indata key=$mykey min=$mykey1 max=$mykey2 | suattributes mode=amp | suop op=db > tmp0
\end{lstlisting}
\par
理论上球面扩散是1/(距离)的平方,但实际上不是需要进行不同的尝试。\par
交互界面能量补偿见igain.sh。\par

\section{单道处理与多道处理区别}
单道处理不用考虑临近道的关系,例如在炮集中和cmp道集中进行处理结果都是一样的。\par
多道处理则是必须考虑道临近道之间的关系。\par
\begin{tabular}{ll}
\toprule
Single-trace & Multi-trace\\
\midrule
sugain – gain & sunmo – NMO\\
sufilter – 1-D filter & sukill – kill trace\\
suvelan – semblance & sudipfilt – 2-D filter\\
supef – deconvolution & sustolt – Stolt migration\\
\bottomrule
\end{tabular}\par

\section{F-K滤波}
$F-K$滤波通过sudipfilt来实现,$F-K$滤波的斜率可以用相对单位(seconds/meter)或者是绝对单位(number of samples)\par
\begin{tabular}{|lll|}
	\toprule
 a pass filter &  slopes=s1,s2,s3,s4  &  amps=0,1,1,0\\

a reject filter  &  slopes=s1,s2,s3,s4  &  amps=1,0,0,1\\
\bottomrule
\end{tabular}\par\mvspace\mvspace
suspecfk查看分析地震数据的频谱:
\begin{lstlisting}
suspecfk < indata.su dx=$dx dt=$dt | suximage xbox=320 ybox=10 wbox=300 hbox=500 \
label1=" Frequency (Hz)" label2="Wavenumber (k)" title="f-k spectrum, no filter"  cmap=hsv2 legend=1 units=Amplitude verbose=0 grid1=dots grid2=dots perc=99 &
\end{lstlisting}
indata.su为需要进行f-k变化的一炮地震数据
\begin{lstlisting}
#通过的数据
sudipfilt < tmp1 dx=$dx dt=$dt slopes=$slopes amps=0,1,1,0 > tmp2
# Plot f-k passed data
suspecfk < tmp2 dx=$dx dt=$dt | 
suximage xbox=320 ybox=10 wbox=300 hbox=500 label1=" Frequency (Hz)" label2="Wavenumber (k)" title="f-k Spectrum after PASS filter" cmap=hsv2 legend=1 units=Amplitude verbose=0 grid1=dots grid2=dots perc=99 &
# Apply reject filter
sudipfilt < tmp1 dx=$dx dt=$dt slopes=$slopes amps=1,0,0,1 > tmp3
# Plot f-k rejected data
suspecfk < tmp3 dx=$dx dt=$dt |
suximage
xbox=940 ybox=10 wbox=300 hbox=500 label1=" Frequency (Hz)" label2="Wavenumber (k)" title="f-k Spectrum after REJECT filter" cmap=hsv2 legend=1 units=Amplitude verbose=0 grid1=dots grid2=dots perc=99 &
\end{lstlisting}

\section{反褶积}
反褶通过压缩子波波形来提高提高时间分辨率。\par
supef (Wiener predictive error filtering)反褶积两个重要的参数是 minlag (seconds)[prediction length] 和 maxlag (seconds)[ operator length]. 
supef parameter pnoise: the percent white noise 白噪音\par
Operator length 一定要大于 prediction length.\par
另个程序suacor画的是每一道的自相关,用来分析反褶积的一个重要工具\par
supef有关于自相关窗口大小的参数设置,而suacor没有自相关窗口参数。\par

\subsection{反褶积前先进行自相关,跟反褶积之后的结果进行对比}
\begin{lstlisting}
#时间截断
suwind < tmp1 tmin=$taa tmax=$tzz > tmp2
#自相关
suacor < tmp2 ntout=101 sym=0 |
suxwigb perc=$myperc xbox=322 ybox=10 wbox=300 hbox=500 label1=" Time (s)" label2=$mykey key=$mykey windowtitle="Acor before decon" title="Autocorrelation" wclip=0 verbose=0 &
\end{lstlisting}

\subsection{反褶积}
\begin{lstlisting}
# Deconvolution
supef < tmp1 minlag=$minlag maxlag=$maxlag mincorr=$taa maxcorr=$tzz pnoise=$wnoise > tmp3
\end{lstlisting}

\subsection{反褶积之后自相关}
\begin{lstlisting}
suwind < tmp3 tmin=$taa tmax=$tzz > tmp4
suacor <tmp4 ntout=101 sym=0 |
suxwigb perc=$myperc xbox=946 ybox=10 wbox=300 hbox=500 label1=" Time (s)" label2=$mykey key=$mykey title="Acor: first time = $taa, last time = $tzz"  windowtitle="Acor after decon" verbose=0 &
\end{lstlisting}

\subsection{先增益还是先滤波}
\begin{lstlisting}
sudiff T5kgf820.su T5kfg820.su > Tdiff820.su
sudiff T5kgf930.su T5kfg930.su > Tdiff930.su
\end{lstlisting}
不知道哪一个更好,需要进行比较,看实际情况而定


\chapter{SU数据绘图及显示}
Seismic Unix软件包包括一部分(普通C语言类型的浮点格式和SU格式)数据绘图的图形工具,均可在X-Windows环境下,在屏幕上显示或生成PostScript格式的图形文件。SU中能够绘制的图形类型有:\\
\begin{tabular}{ll}
	\toprule
	contour plots & 等值线绘图\\	
	gray or colorscale image plots & 灰度或彩色图像\\	
	wiggle trace plots & 地震道波形图	\\
	line or symbol graphs & 测线和符号绘图\\
	movies & 电影动画\\
	3D cube plots (PostScript only) & 3D立体图\\
	\bottomrule
\end{tabular}

\section{X-Windows中绘制一般浮点型数据}
在X-Windows环境下,用于绘制浮点型数据(没有SU道头)的程序有:\\
\begin{tabular}{ll}
	\toprule
	xcontour & 调用矢量绘图来绘制$f(x1,x2)$的等值线图\\
	ximage & 绘制灰度或彩色图像\\
	xwigb & 绘制变面积地震道波形图\\
	xgraph & 绘制曲线\\
	xmovie & 绘制动画电影\\
	\bottomrule
\end{tabular}\\\\
例子:
\begin{lstlisting}
suplane | sustrip > data.bin
xcontour < data.bin n1=64 n2=32 title="contour" 
ximage < data.bin n1=64 n2=32 title="image" 
xwigb < data.bin n1=64 n2=32 title="wiggle trace" 
xmovie < data.bin n1=64 n2=32 title="movie"
xgraph < data.bin n=5 
\end{lstlisting}
上面命令中$n1=64,n2=32$表示数据大小为:$64*32$。\\
完成上述操作后,要消去窗口,可以点击窗口然后键入“q”即可退出。

\section{X-Windows中SU格式数据绘图}
相应,在X-Windows环境下,用于绘制SU格式数据的程序有:\\
\begin{tabular}{ll}
	\toprule
	suxcontour & 调用矢量绘图来绘制f(x1,x2)的等值线图\\
	suximage & 绘制灰度或彩色图像\\
	suxwigb & 绘制变面积地震道波形图\\
	suxgraph & 绘制曲线\\
	suxmovie & 绘制动画电影\\
	suxmax & SU数据每道最大值、最小值和最大绝对值X-Windwos图\\
	\bottomrule
\end{tabular}\\\\
例子:
\begin{lstlisting}
suplane | suxcontour title="contour" 
suplane | suximage title="image" 
suplane | suxwigb title="wiggle trace" 
suplane | suxgraph title="graph" 
suplane | suxmovie title="movie" 
suplane | suxmax title="max" 
\end{lstlisting}
绘制地震道时,使用真偏移距,需要使用“key=”关键字参数。例如:
\begin{lstlisting}
suplane | suchw key1=offset key2=tracl a=0 b=100 | suxwigb key=offset
\end{lstlisting}
用suxmovie命令制作电影动画:
\begin{lstlisting}
suplane > junk1.su
suplane | suaddnoise sn=20 >> junk1.su
suplane | suaddnoise sn=15 >> junk1.su
suplane | suaddnoise sn=10 >> junk1.su
suplane | suaddnoise sn=5 >> junk1.su
suplane | suaddnoise sn=3 >> junk1.su
suplane | suaddnoise sn=2 >> junk1.su
suplane | suaddnoise sn=1 >> junk1.su
suxmovie < junk1.su n2=32 title="frame=%g" loop=1 
\end{lstlisting}
这里参数n2=32表示数据每幅有32道。“\%g”用来在图形标题上显示出图形的桢数,参数“loop=1”表示用连续循环方式运行电影。\par
要加快或减慢电影的放映速度,只需要在图形的右下角点击并拖动窗口来放大或缩小图像。按在鼠标最右边的键一次将暂停,按第二次将重新放映。

\section{一般浮点型数据的PostScript绘图}
可用于一般浮点型数据的PostScript绘图的命令有:\\
\begin{tabular}{lp{0.8\textwidth}}
	\toprule
pscontour & PostScript CONTOURing of a two-dimensional function$f(x1,x2)$\\
psimage & PostScript IMAGE plot of a uniformly-sampled function $f(x1,x2)$\\
pscube & PostScript image plot of a data CUBE\\
psgraph & PostScript GRAPHer Graphs $n[i]$ pairs of $(x,y)$ coordinates\\
psmovie & PostScript MOVIE plot of a uniformly-sampled function $f(x1,x2,x3)$\\
pswigb & PostScript WIGgle-trace plot of $f(x1,x2)$ via Bitmap, 用位图绘制地震道\\
pswigp & PSWIGP - PostScript WIGgle-trace plot of $f(x1,x2)$ via Polygons. 用多边形绘制地震道\\
\bottomrule
\end{tabular}\\\\
例子:
\begin{lstlisting}
suplane | sustrip > data.bin
pscontour < data.bin n1=64 n2=32 title="contour" > data1.eps
psimage < data.bin n1=64 n2=32 title="image" > data2.eps
pscube < data.bin n1=64 n2=32 title="cube plot" > data4.eps
pswigb < data.bin n1=64 n2=32 title="bitmap wiggle trace" > data3.eps
pswigp < data.bin n1=64 n2=32 title="wiggle trace" > data4.eps
psmovie < data.bin n1=64 n2=32 title="movie" > data5.eps

a2b < data.ascii n1=2 > data.bin
n=5

psgraph < data.bin n=5 > data6.eps
\end{lstlisting}

\section{SU数据的PostScript绘图}
相应,可用于SU数据的PostScript绘图命令有:\par
\begin{tabular}{lp{0.7\textwidth}}
	\toprule
	supscontour & PostScript CONTOUR plot of an SU data set\\
	supsimage & PostScript IMAGE plot of an SU data set\\
	supscube & PostScript CUBE plot of an SU data set\\
	supsgraph & PostScript GRAPH plot of an SU data set\\
	supswigb & PostScript Bit-mapped WIGgle plot of an SU data set\\
	supswigp & PostScript Polygon-filled WIGgle plot of an SU data set\\
	supsmax & PostScript of the MAX, min, or absolute max value on each trace of a SU data set\\
	\bottomrule
\end{tabular}\\\\
例子:
\begin{lstlisting}
suplane > junk.su 
supscontour < junk.su title="contour" > data1.eps
supsimage < junk.su title="image" label1="sec" label2="trace number" > data2.eps
supscube < junk.su title="cube plot" > data4.eps
supswigb < junk.su title="bitmap wiggle trace" > data3.eps
supswigp < junk.su title="wiggle trace" > data4.eps
supsmovie < junk.su title="movie" > data5.eps
supsmax < junk.su title="max" > data5.eps
\end{lstlisting}

\section{其它PostScript绘图工具}
\begin{tabular}{lp{0.7\textwidth}}
	\toprule
	psbbox & change BoundingBOX of existing PostScript file 改变PostScript文件四周的框架\\
	psmerge & MERGE PostScript files 合并PostScript文件\\
	merge2 & MERGE2 PostScript figures onto one page 合并两幅图像\\
	merge4 & MERGE4 figures onto one page 合并四幅图像\\
	pslabel & output PostScript file consisting of a single TEXT string on a specified background.(Use with psmerge to label plots.) \\
	psmanager & printer MANAGER for HP 4MV and HP 5Si Mx Laserjet PostScript printing\\
	psepsi & add an EPSI formatted preview bitmap to an EPS file\\
	\bottomrule
\end{tabular}\\\\
例子:
\begin{lstlisting}
suplane > junk.su
suplane | sufilter > junk1.su
supswigb < junk.su title="Wiggle trace" label1="sec" label2="trace number" > junk1.eps
supsimage < junk.su title="Image Plot" label1="sec" label2="trace number" > junk2.eps
supscontour < junk.su title="Contour Plot" label1="sec" label2="trace number" > junk3.eps
supswigb < junk1.su title="Filtered" label1="sec" label2="trace number" > junk4.eps
\end{lstlisting}
合并两幅图到一个文件中去:
\begin{lstlisting}
merge2 junk1.eps junk2.eps > junk.m2.eps
\end{lstlisting}
合并4幅图到一个文件中去:
\begin{lstlisting}
merge2 junk1.eps junk2.eps junk3.eps junk4.eps > junk.m4.eps
\end{lstlisting}